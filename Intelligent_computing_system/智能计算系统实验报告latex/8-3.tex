%!TEX program = xelatex
% 完整编译: xelatex -> biber/bibtex -> xelatex -> xelatex
\documentclass[lang=cn,a4paper]{shu-lab-report}


% 本文档命令
\usepackage{array}
\newcommand{\ccr}[1]{\makecell{{\color{#1}\rule{1cm}{1cm}}}}
\renewcommand{\lstlistingname}{代码}




\begin{document}

%%%%%%%%%%%%%%%%%%%%%%%%%%%%%
%% The Cover Page
%%%%%%%%%%%%%%%%%%%%%%%%%%%%%

\thispagestyle{empty}

~\\


\begin{figure}[!htbp]
    \centering
    \includegraphics[width=10.5cm]{image/shulogo.png}
\end{figure}


\vspace{12mm}

\centerline{\kaishu\Huge{2024-2025学年冬季学期}}

\vspace{6mm}

\centerline{\kaishu\Huge{《智能计算系统》(08696037)}}

\vspace{6mm}

\centerline{\kaishu\Huge{实验报告}}


\vspace{16mm}

\begin{center}
\renewcommand\arraystretch{1.5}
\begin{tabular}{r c}
    \makebox[8em][s]{\LARGE{姓名}}    & \LARGE{汪江豪}\\  \cmidrule(l){2-2} 
    \makebox[8em][s]{\LARGE{学号}}    & \LARGE{22121630}\\  \cmidrule(l){2-2} 
    \makebox[8em][s]{\LARGE{实验名称}} & \LARGE{基于 Code Llama 实现代码生成}\\     \cmidrule(l){2-2} 
    \makebox[8em][s]{\LARGE{日期}}    & \LARGE{2025年2月8日}\\ \cmidrule(l){2-2}
\end{tabular}
\end{center}

\vspace{8mm}

\begin{table}[!htp]
        \renewcommand{\arraystretch}{1.5} % 仅影响此表的行高,行高设置为默认的1.5倍
	\centering
	\begin{tabular}{|c|c|c|c|c|}
		\hline
		\multirow{2}{*}{\Large \textbf{实验 (70分)}} & \Large 目标1 & \Large 目标2 & \Large 目标3 & \Large 得分 \\ \cline{2-5}
		                                    &             &             &             &            \\ \hline
		\multirow{2}{*}{\Large \textbf{报告 (30分)}} & \Large 代码 (10分) & \Large 结果 (10分) & \Large 格式 (10分) & \Large 得分 \\ \cline{2-5}
		                                    &             &             &             &            \\ \hline
		\multicolumn{2}{|c|}{\Large \textbf{批阅人}} & \multicolumn{2}{c|}{\Large \textbf{批阅日期}} & \Large \textbf{总得分} \\ \hline
		\multicolumn{2}{|c|}{}              & \multicolumn{2}{c|}{}              &            \\ \cline{1-2} \cline{3-5}
	\end{tabular}
\end{table}

\vspace{6mm}

\centerline{\kaishu\Large{上海大学\ 计算机工程与科学学院}}


%%%%%%
\newpage
\hypersetup{linkcolor=black,citecolor=black}
\tableofcontents
%%%%%%

%%%%%%%%%%%%%%%%%%%%%%%%%%%%%
%% The Content
%%%%%%%%%%%%%%%%%%%%%%%%%%%%%

\newpage
\setcounter{page}{1}


\section{选修实验8-3 基于 Code Llama 实现代码生成}

\subsection{实验目的}
\begin{itemize}
\item[$\bullet$] 本实验旨在介绍基于Code Llama实现代码生成的基本原理和操作步骤;
\item[$\bullet$] 学生将深入了解基于Code Llama实现代码生成关键步骤,包括基于transformers库的Codel Llama推理模块、Transformer模型构建模块、Llama代码生成模块等;
\item[$\bullet$] 掌握如何在 DLP 平台上部署基于Code Llama的代码生成模型,能够实现Code Llama 代码生成模块、Code Llama 代码补全模块、Code Llama 指令微调模块。
\end{itemize}


\subsection{实验环境}
\begin{itemize}
    \item[$\bullet$] 硬件平台:DLP云平台环境。
    \item[$\bullet$] 软件环境:编程框架Pytorch1.13.1、CNNL高性能AI运算库,CNRT 运行时库,以及 python 环境及相关的扩展库。
\end{itemize}

\subsection{评分标准}
\begin{itemize}
\item[$\bullet$] 60分标准:能够正确实现基于transformers库的Codel Llama推理模块。
\item[$\bullet$] 70分标准:在60分标准基础上,能够正确实现Transformer模型构建模块、Llama代码生成模块。
\item[$\bullet$] 80分标准:在70分标准基础上,能够正确实现Code Llama代码生成模块。
\item[$\bullet$] 90分标准:在80分标准基础上,能够正确实现Code Llama 代码补全模块。
\item[$\bullet$] 100分标准:在90分标准基础上,能够正确实现Code Llama 指令微调模块。
\end{itemize}

\subsection{实验内容及步骤}
\subsubsection{补全llama\_mlu/model.py}
\par 在该文件中,ModelArgs定义了模型的参数。
RMSNorm实现了RMS归一化层,用于对输入张量进行归一化处理等。
\par 该文件主要定义了一个Transformer模型的结构和功能,包括模型参数、归一化层、多头注意力机制、
前馈神经网络层、Transformer块以及整个Transformer模型的实现。
还包括了一些辅助函数,用于处理张量的形状和频率嵌入。
\par 由于篇幅限制,补全详情见代码

\subsubsection{补全llama\_mlu/generation.py}
\par 给该文件主要实现了基于Llama模型的文本生成系统,包括文本补全、文本填充和对话生成等多种任务。
通过定义Llama类封装了模型构建、参数加载、分布式初始化以及生成过程。
同时提供了辅助函数(如sample\_top\_p、infilling\_prompt\_tokens和dialog\_prompt\_tokens)
用于处理采样、提示编码和对话格式化,从而支持多种生成场景。
\par 由于该文件是实现推理的核心功能,我将补全代码部分展示出。

\begin{lstlisting}[language=python, caption={build函数}, label={code1}]
    class Llama:
    @staticmethod
    def build(
        ckpt_dir: str,
        tokenizer_path: str,
        max_seq_len: int,
        max_batch_size: int,
        model_parallel_size: Optional[int] = None,
    ) -> "Llama":
        if not torch.distributed.is_initialized():
            if device == "mlu":
                # TODO: 使用 MLU 设备初始化分布式进程组
                torch.distributed.init_process_group("cncl")
            else:
                torch.distributed.init_process_group("gloo")
        if not model_parallel_is_initialized():
            if model_parallel_size is None:
                model_parallel_size = int(os.environ.get("WORLD_SIZE", 1))
            initialize_model_parallel(model_parallel_size)

        local_rank = int(os.environ.get("LOCAL_RANK", 0))
        if device == "mlu":
            # TODO: 如果设备为 MLU,则设置当前进程的 MLU 设备
            torch.mlu.set_device(local_rank)

        # seed must be the same in all processes
        torch.manual_seed(1)

        if local_rank > 0:
            sys.stdout = open(os.devnull, "w")

        start_time = time.time()
        checkpoints = sorted(Path(ckpt_dir).glob("*.pth"))
        assert len(checkpoints) > 0, f"no checkpoint files found in {ckpt_dir}"
        assert model_parallel_size == len(
            checkpoints
        ), f"Loading a checkpoint for MP={len(checkpoints)} but world size is {model_parallel_size}"
        ckpt_path = checkpoints[get_model_parallel_rank()]
        # TODO:加载模型的检查点文件,并将模型加载到 CPU 上。
        checkpoint = torch.load(ckpt_path, map_location="cpu")
        with open(Path(ckpt_dir) / "params.json", "r") as f:
            params = json.loads(f.read())

        model_args: ModelArgs = ModelArgs(
            max_seq_len=max_seq_len,
            max_batch_size=max_batch_size,
            **params,
        )
        # TODO: 调用Tokenizer函数
        tokenizer = Tokenizer(tokenizer_path)
        model_args.vocab_size = tokenizer.n_words
        # support for mac
        # print(device)
        if device == "mlu":
            torch.set_default_tensor_type(torch.HalfTensor)
        else:
            torch.set_default_tensor_type(torch.HalfTensor)
        # TODO: 调用Transformer 模型
        model = Transformer(model_args)
        # TODO:加载模型的参数字典
        model.load_state_dict(checkpoint, strict=False)
        print("TRANSFORMER MODEL PASS!")
        # add start
        # print(device)
        if device == "cpu":
            model = model.float()

        # add end
        model.to(device)
        print(f"Loaded in {time.time() - start_time:.2f} seconds")

        print("LLAMA BUILD PASS!")
        return Llama(model, tokenizer)
\end{lstlisting}

\par 该build函数主要是构建并初始化Llama模型,包括初始化分布式进程组,
根据环境变量设置本地设备,加载指定检查点目录下的预训练模型和模型配置,
根据检查点文件数与模型并行大小进行校验,调用Tokenizer加载分词器,根据分词器设置模型词汇表大小,
构建Transformer模型,并加载检查点中的权重,最后将模型转移到指定设备上。
返回封装了模型与分词器的Llama生成器实例。

\begin{lstlisting}[language=python, caption={generate函数}, label={code2}]
        def generate(
        self,
        prompt_tokens: List[List[int]],
        max_gen_len: int,
        temperature: float = 0.6,
        top_p: float = 0.9,
        logprobs: bool = False,
        echo: bool = False,
        stop_token: Optional[int] = None,
    ) -> Tuple[List[List[int]], Optional[List[List[float]]]]:
        # print("调用了generate")
        if stop_token is None:
            stop_token = self.tokenizer.eos_id
        params = self.model.params
        bsz = len(prompt_tokens)
        assert bsz <= params.max_batch_size, (bsz, params.max_batch_size)
        # TODO:  获取提示文本序列中最短的长度
        min_prompt_len = min(len(t) for t in prompt_tokens)
        # TODO: 获取提示文本序列中最长的长度
        max_prompt_len = max(len(t) for t in prompt_tokens)
        assert max_prompt_len <= params.max_seq_len
        # TODO: 计算生成的总长度,需考虑提示文本和最大生成长度
        total_len = max_prompt_len + max_gen_len
        pad_id = self.tokenizer.pad_id
        tokens = torch.full((bsz, total_len), pad_id, dtype=torch.long, device=device)
        for k, t in enumerate(prompt_tokens):
            # TODO:将提示文本编码添加到张量中
            tokens[k, : len(t)] = torch.tensor(t, dtype=torch.long, device=device)
        if logprobs:
            # TODO:创建一个与tokens张量具有相同形状的全零张量
            token_logprobs = torch.zeros_like(tokens, dtype=torch.float)
        prev_pos = 0
        stop_reached = torch.tensor([False] * bsz, device=device).to(device)
        input_text_mask = tokens != pad_id
        # print("generate第189行")
        # k = 1
        # print(f"min_prompt_len={min_prompt_len},total_len={total_len}")
        for cur_pos in range(min_prompt_len, total_len):
            logits = self.model.forward(tokens[:, prev_pos:cur_pos], prev_pos)
            # print(f"开始{k}次for循环")
            if logprobs:
                token_logprobs[:, prev_pos + 1 : cur_pos + 1] = -F.cross_entropy(
                    input=logits.transpose(1, 2),
                    target=tokens[:, prev_pos + 1 : cur_pos + 1],
                    reduction="none",
                    ignore_index=pad_id,
                )
            if temperature > 0:
                # TODO: 对模型的输出进行 softmax 归一化,以得到每个可能的下一个token的概率分布,其中 temperature 用于控制模型输出的多样性
                probs = F.softmax(logits[:, -1, :] / temperature, dim=-1)
                # TODO: 根据概率分布采样出下一个 token
                next_token = torch.multinomial(probs, num_samples=1)
            else:
                # TODO:直接选择logits最大的位置作为下一个token,不进行随机采样
                next_token = torch.argmax(logits[:, -1, :], dim=-1, keepdim=True)
            next_token = next_token.reshape(-1)
            # only replace token if prompt has already been generated
            next_token = torch.where(
                input_text_mask[:, cur_pos], tokens[:, cur_pos], next_token
            )
            tokens[:, cur_pos] = next_token
            stop_reached |= (~input_text_mask[:, cur_pos]) & (next_token == stop_token)
            prev_pos = cur_pos
            # print(f"经历了{k}次for循环")
            # k += 1
            if all(stop_reached):
                break
        # print("for循环结束了")
        if logprobs:
            # TODO: 将张量转换为列表格式
            token_logprobs = token_logprobs.tolist()
        out_tokens, out_logprobs = [], []
        for i, toks in enumerate(tokens.tolist()):
            # cut to max gen len
            start = 0 if echo else len(prompt_tokens[i])
            # TODO: 截取生成的标记序列,直到达到最大生成长度
            toks = toks[start : start + max_gen_len]
            probs = None
            if logprobs:
                probs = token_logprobs[i][start : len(prompt_tokens[i]) + max_gen_len]
            # cut to stop token if present
            if stop_token in toks:
                stop_idx = toks.index(stop_token)
                toks = toks[:stop_idx]
                probs = probs[:stop_idx] if logprobs else None
            # TODO: 将截取后的标记序列添加到输出列表中
            out_tokens.append(toks)
            # TODO: 将截取后的log概率列表添加到输出列表中
            if logprobs:
                out_logprobs.append(probs)
        print("LLAMA GENERATE PASS!")
        return (out_tokens, out_logprobs if logprobs else None)
\end{lstlisting}

\par 该generate函数负责根据输入的提示文本,迭代生成后续token,直到达到最大生成长度或遇到停止token为止。
其流程是:对输入的prompt\_tokens进行padding,并设置生成序列的总长度。
循环调用模型的forward方法,基于当前生成的序列预测下一个token。
根据温度参数对模型输出进行softmax归一化,然后采样得到下一个token(或直接取最大概率token)。
使用mask确保已存在的prompt token不被覆盖,并检查是否达到停止条件,最后将生成的token序列(以及可选的log概率)截取后返回。

\begin{lstlisting}[language=python, caption={text\_completion函数}, label={code3}]
        def text_completion(
        self,
        prompts: List[str],
        temperature: float = 0.6,
        top_p: float = 0.9,
        max_gen_len: Optional[int] = None,
        logprobs: bool = False,
        echo: bool = False,
    ) -> List[CompletionPrediction]:
        if max_gen_len is None:
            max_gen_len = self.model.params.max_seq_len - 1
        prompt_tokens = [
            self.tokenizer.encode(str(x), bos=True, eos=False) for x in prompts
        ]
        # TODO: 调用 generate 方法生成文本
        generation_tokens, generation_logprobs = self.generate(
            prompt_tokens, max_gen_len, temperature, top_p, logprobs, echo
        )
        if logprobs:
            assert generation_logprobs is not None
            return [
                {
                    "generation": self.tokenizer.decode(t),
                    "tokens": [self.tokenizer.token_piece(x) for x in t],
                    "logprobs": logprobs_i,
                }
                for t, logprobs_i in zip(generation_tokens, generation_logprobs)
            ]
        print("LLAMA TEXTCOMPLETION PASS!")
        return [{"generation": self.tokenizer.decode(t)} for t in generation_tokens]
\end{lstlisting}

\par 该函数实现了文本补全任务的具体流程:
首先将输入的字符串提示使用分词器进行编码生成token序列;
然后调用generate函数基于这些token序列生成后续文本,
生成序列长度由max\_gen\_len决定;
最后根据是否需要返回log概率,将生成的token序列解码为文本输出,
并包装成包含生成文本、token碎片和log概率(可选)的字典列表返回。

\begin{lstlisting}[language=python, caption={text\_infilling函数}, label={code4}]
        def text_infilling(
        self,
        prefixes: List[str],
        suffixes: List[str],
        temperature: float = 0.6,
        top_p: float = 0.9,
        max_gen_len: Optional[int] = None,
        logprobs: bool = False,
        suffix_first: bool = False,
    ) -> List[InfillingPrediction]:
        assert self.tokenizer.eot_id is not None
        # print("调用了text_infilling")
        if max_gen_len is None:
            max_gen_len = self.model.params.max_seq_len - 1
        prompt_tokens = [
            # TODO: 调用函数对每个前缀和后缀进行处理,生成填充问题的编码
            infilling_prompt_tokens(self.tokenizer, prefix, suffix, suffix_first)
            for prefix, suffix in zip(prefixes, suffixes)
        ]
        # TODO:调用 generate 方法生成文本
        generation_tokens, generation_logprobs = self.generate(
            prompt_tokens,
            max_gen_len,
            temperature,
            top_p,
            logprobs,
        )

        generations = [self.tokenizer.decode_infilling(t) for t in generation_tokens]
        print("LLAMA TEXTINFILLING PASS!")
        if logprobs:
            assert generation_logprobs is not None
            return [
                {
                    "generation": generation,
                    "logprobs": logprobs_i,
                    "tokens": [self.tokenizer.token_piece(x) for x in t],
                    "full_text": prefix + generation + suffix,
                }
                for prefix, suffix, generation, t, logprobs_i in zip(
                    prefixes,
                    suffixes,
                    generations,
                    generation_tokens,
                    generation_logprobs,
                )
            ]
        else:
            return [
                {
                    "generation": generation,
                    "full_text": prefix + generation + suffix,
                }
                for prefix, suffix, generation in zip(prefixes, suffixes, generations)
            ]
\end{lstlisting}

\par 该函数用于在文本中插入缺失内容。
首先将前缀和后缀编码为提示序列,
然后调用generate函数生成中间填充文本,
最后将生成的文本与前后缀拼接并返回,
可选择输出每个token的log概率信息。

\begin{lstlisting}[language=python, caption={chat\_completion函数}, label={code5}]
        def chat_completion(
        self,
        dialogs: List[Dialog],
        temperature: float = 0.6,
        top_p: float = 0.9,
        max_gen_len: Optional[int] = None,
        logprobs: bool = False,
    ) -> List[ChatPrediction]:
        if self.tokenizer.step_id is not None:
            ## 如果模型支持 step_id,则使用另一种chat_completion的方法
            return self._chat_completion_with_step_id(dialogs, temperature, top_p, max_gen_len, logprobs)
        if max_gen_len is None:
            max_gen_len = self.model.params.max_seq_len - 1
        prompt_tokens = []
        unsafe_requests = []
        for dialog in dialogs:
            unsafe_requests.append(
                any([tag in msg["content"] for tag in SPECIAL_TAGS for msg in dialog])
            )
            if dialog[0]["role"] == "system":
                dialog = [  # type: ignore
                    {
                        "role": dialog[1]["role"],
                        "content": B_SYS
                        + dialog[0]["content"]
                        + E_SYS
                        + dialog[1]["content"],
                    }
                ] + dialog[2:]
            assert all([msg["role"] == "user" for msg in dialog[::2]]) and all(
                [msg["role"] == "assistant" for msg in dialog[1::2]]
            ), (
                "model only supports 'system', 'user' and 'assistant' roles, "
                "starting with 'system', then 'user' and alternating (u/a/u/a/u...)"
            )
            dialog_tokens: List[int] = sum(
                [
                    self.tokenizer.encode(
                        f"{B_INST} {prompt['content'].strip()} {E_INST} {answer['content'].strip()} ",
                        bos=True,
                        eos=True,
                    )
                    for prompt, answer in zip(
                        dialog[::2],
                        dialog[1::2],
                    )
                ],
                [],
            )
            assert (
                dialog[-1]["role"] == "user"
            ), f"Last message must be from user, got {dialog[-1]['role']}"
            dialog_tokens += self.tokenizer.encode(
                f"{B_INST} {dialog[-1]['content'].strip()} {E_INST}",
                bos=True,
                eos=False,
            )
            prompt_tokens.append(dialog_tokens)
        # TODO:调用 generate 方法生成文本
        generation_tokens, generation_logprobs = self.generate(
            prompt_tokens,
            max_gen_len,
            temperature,
            top_p,
            logprobs,
        )
        print("LLAMA CHATCOMPLETION PASS!")
        if logprobs:
            assert generation_logprobs is not None
            return [
                {
                    "generation": {  # type: ignore
                        "role": "assistant",
                        "content": (
                            self.tokenizer.decode(t) if not unsafe else UNSAFE_ERROR
                        ),
                    },
                    "tokens": [self.tokenizer.token_piece(x) for x in t],
                    "logprobs": logprobs_i,
                }
                for t, logprobs_i, unsafe in zip(
                    generation_tokens, generation_logprobs, unsafe_requests
                )
            ]
        return [
            {
                "generation": {  # type: ignore
                    "role": "assistant",
                    "content": self.tokenizer.decode(t) if not unsafe else UNSAFE_ERROR,
                }
            }
            for t, unsafe in zip(generation_tokens, unsafe_requests)
        ]
\end{lstlisting}

\par 此函数用于生成聊天对话的补全内容。
它会将输入的对话消息转换为提示tokens,
调用generate函数生成回复,然后将生成的结果解码为文本返回。
它还会检测对话中是否包含非法标记,遇到这些内容则输出错误提示。

\begin{lstlisting}[language=python, caption={\_chat\_completion\_turns函数}, label={code6}]
        def _chat_completion_turns(
        self,
        dialogs: List[Dialog],
        temperature: float = 0.6,
        top_p: float = 0.9,
        max_gen_len: Optional[int] = None,
        logprobs: bool = False,
    ) -> List[ChatPrediction]:
        if self.tokenizer.step_id is None:
            raise RuntimeError("Model not suitable for chat_completion_step()")
        if max_gen_len is None:
            max_gen_len = self.model.params.max_seq_len - 1

        prompt_tokens = []
        unsafe_requests = []
        for dialog in dialogs:
            unsafe_requests.append(
                any([tag in msg["content"] for tag in SPECIAL_TAGS for msg in dialog])
            )

            # Insert system message if not provided
            if dialog[0]["role"] != "system":
                dialog = [{"role": "system", "content": ""}] + dialog  # type: ignore
            # TODO:调用函数将对话格式化为模型可处理的对话提示编码
            dialog_tokens = dialog_prompt_tokens(self.tokenizer, dialog)
            prompt_tokens.append(dialog_tokens)
        # TODO:调用 generate 方法生成文本
        generation_tokens, generation_logprobs = self.generate(
            prompt_tokens,
            max_gen_len,
            temperature,
            top_p,
            logprobs,
        )
        if logprobs:
            assert generation_logprobs is not None
            return [
                {
                    "generation": {
                        "role": "assistant",
                        "destination": "user",
                        "content": (
                            self.tokenizer.decode(t) if not unsafe else UNSAFE_ERROR
                        ),
                    },
                    "tokens": [self.tokenizer.token_piece(x) for x in t],
                    "logprobs": logprobs_i,
                }
                for t, logprobs_i, unsafe in zip(
                    generation_tokens, generation_logprobs, unsafe_requests
                )
            ]
        return [
            {
                "generation": {
                    "role": "assistant",
                    "destination": "user",
                    "content": self.tokenizer.decode(t) if not unsafe else UNSAFE_ERROR,
                }
            }
            for t, unsafe in zip(generation_tokens, unsafe_requests)
        ]
\end{lstlisting}

\par 该函数用于在支持step\_id的场景下生成多轮对话的回复:
如果对话中缺少system角色,会自动插入一条空的system消息。
调用dialog\_prompt\_tokens将对话转为可处理的token序列。
使用generate函数生成后续内容,解码成文本并封装为JSON结构返回。
若对话包含非法标记,则输出相应错误提示。

\subsubsection{补全Codellama-inference.py}

% \begin{itemize}
% \item[$\bullet$]\textbf{编译环境配置}
% \end{itemize}

\par 该文件主要功能是测试Code Llama模型在代码补全任务中的性能。
\par 由于篇幅限制,补全详情见代码
\par 运行\ref{shell1}以测试代码效果

\begin{lstlisting}[language=bash, caption={run-cll.sh}, label={shell1}]
export MLU_VISIBLE_DEVICES='0,1'
python Codellama-infer.py
\end{lstlisting}

\par 测试效果如图\ref{fig:8-3-1}所示。
\par 值得注意的是,在运行过程中,不断提示寒武纪的mlu不支持64位的数据类型,因此需要将代码中的数据类型改为32位。

\begin{figure}[!htbp]
    \centering
    \includegraphics[width=1\textwidth]{image/8-3/8-3-1.png}
    \caption{Codellama-inference.py测试结果}
    \label{fig:8-3-1}
\end{figure}

\subsubsection{补全example\_completion\_mlu.py}
\par 该文件主要功能是使用Llama模型进行文本生成任务,使用了generate.py文件中Llama类的text\_completion方法。
\par 由于篇幅限制,补全详情见代码
\par 运行\ref{shell2}脚本后,能显示测试结果。

\begin{lstlisting}[language=bash, caption={run\_completion.sh}, label={shell2}]
torchrun --nproc_per_node 1 example_completion_mlu.py \
    --ckpt_dir /workspace/model/favorite/large-scale-models/model-v1/CodeLlama-7b/ \
    --tokenizer_path /workspace/model/favorite/large-scale-models/model-v1/CodeLlama-7b/tokenizer.model \
    --max_seq_len 128 --max_batch_size 4
\end{lstlisting}

\par 测试结果如图\ref{fig:8-3-2}所示。

\begin{figure}[!htbp]
    \centering
    \includegraphics[width=1\textwidth]{image/8-3/8-3-2.png}
    \caption{example\_completion\_mlu.py测试结果}
    \label{fig:8-3-2}
\end{figure}

\par 内容生成逻辑正确。个人猜测可能由于generate.py中,对应的文本推理方法中,默认参数没有调整,也没有重复惩罚措施,导致生成的文本出现重复内容。


\subsubsection{补全example\_infilling\_mlu.py}
\par 该文件使用Llama模型进行文本填充任务,通过加载模型分词器,处理包含占位符的提示语,
然后使用Llama模型生成填充文本,该功能主要使用了generate.py文件中Llama类的text\_infilling方法。
\par 由于篇幅限制,补全详情见代码
\par 运行\ref{shell3}脚本后,能显示测试结果。

\begin{lstlisting}[language=bash, caption={run\_completion.sh}, label={shell3}]
torchrun --nproc_per_node 1 example_infilling_mlu.py \
    --ckpt_dir /workspace/model/favorite/large-scale-models/model-v1/CodeLlama-7b/ \
    --tokenizer_path /workspace/model/favorite/large-scale-models/model-v1/CodeLlama-7b/tokenizer.model \
    --max_seq_len 192 --max_batch_size 4
\end{lstlisting}

\par 运行后结果如图\ref{fig:8-3-3}。

\begin{figure}[!htbp]
    \centering
    \includegraphics[width=1\textwidth]{image/8-3/8-3-3.png}
    \caption{example\_infilling\_mlu.py测试结果}
    \label{fig:8-3-3}
\end{figure}

\subsubsection{补全example\_instructions\_mlu.py}
\par 该文件使用Llama模型生成对话文本,同故宫加载模型和分词器,处理用户提供的指令,
使用Llama模型生成相应的回复,并打印完整的对话内容,该功能主要使用了generate.py文件中Llama类的chat\_completion方法。
\par 由于篇幅限制,补全详情见代码
\par 运行\ref{shell4}脚本后,能显示测试结果。

\begin{lstlisting}[language=bash, caption={run\_completion.sh}, label={shell4}]
torchrun --nproc_per_node 1 example_instructions_mlu.py \
    --ckpt_dir /workspace/model/favorite/large-scale-models/model-v1/CodeLlama-7b-Instruct/ \
    --tokenizer_path /workspace/model/favorite/large-scale-models/model-v1/CodeLlama-7b-Instruct/tokenizer.model \
    --max_seq_len 512 --max_batch_size 4
\end{lstlisting}

\par 运行后结果如图\ref{fig:8-3-4}。

\begin{figure}[!htbp]
    \centering
    \includegraphics[width=1\textwidth]{image/8-3/8-3-4.png}
    \caption{example\_instructions\_mlu.py测试结果}
    \label{fig:8-3-4}
\end{figure}

\par 可见,代码正确跑出结果,但还是由于方法中的temperature、top\_p等默认参数没有调整,或所给的模型缺乏重复惩罚措施,导致生成的文本出现重复内容。
\par 给出希冀平台评测结果如图\ref{fig:8-3}。

\begin{figure}[!htbp]
    \centering
    \includegraphics[width=1\textwidth]{image/8-3/8-3-final.png}
    \caption{8-3测试结果}
    \label{fig:8-3}
\end{figure}

\par 可见,实验结果符合预期,实验完成。

\subsection{实验总结}
\par 实验初期,面对复杂的实验任务和众多的代码文件,我感到有些不知所措。
但通过仔细阅读代码,逐步理解代码的功能和结构,我逐渐了解了一些代码的作用和实现原理。
这些都促使我不断思考,不断尝试,最终完成了实验任务。
\par 虽然本次实验取得了一定的成果,但也暴露出了一些不足之处。
例如模型生成的质量可能并不十全十美,我也意识到大模型的开发过程中的复杂性和困难性。
因此,我会继续努力,不断学习,提高自己的技术水平,为未来的研究和实践打下坚实的基础。

\end{document}
