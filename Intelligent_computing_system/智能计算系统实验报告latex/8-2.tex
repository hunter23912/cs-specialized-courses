%!TEX program = xelatex
% 完整编译: xelatex -> biber/bibtex -> xelatex -> xelatex
\documentclass[lang=cn,a4paper]{shu-lab-report}


% 本文档命令
\usepackage{array}
\newcommand{\ccr}[1]{\makecell{{\color{#1}\rule{1cm}{1cm}}}}
\renewcommand{\lstlistingname}{代码}
% \titleformat{\section}{block}
% {\centering}{}{}{}

\begin{document}

%%%%%%%%%%%%%%%%%%%%%%%%%%%%%
%% The Cover Page
%%%%%%%%%%%%%%%%%%%%%%%%%%%%%

\thispagestyle{empty}

~\\


\begin{figure}[!htbp]
    \centering
    \includegraphics[width=10.5cm]{image/shulogo.png}
\end{figure}


\vspace{12mm}

\centerline{\kaishu\Huge{2024-2025学年冬季学期}}

\vspace{6mm}

\centerline{\kaishu\Huge{《智能计算系统》(08696037)}}

\vspace{6mm}

\centerline{\kaishu\Huge{实验报告}}


\vspace{16mm}

\begin{center}
\renewcommand\arraystretch{1.5}
\begin{tabular}{r c}
    \makebox[8em][s]{\LARGE{姓名}}    & \LARGE{汪江豪}\\  \cmidrule(l){2-2} 
    \makebox[8em][s]{\LARGE{学号}}    & \LARGE{22121630}\\  \cmidrule(l){2-2} 
    \makebox[8em][s]{\LARGE{实验名称}} & \LARGE{基于Llama 2实现聊天机器人}\\     \cmidrule(l){2-2} 
    \makebox[8em][s]{\LARGE{日期}}    & \LARGE{2025年2月8日}\\ \cmidrule(l){2-2}
\end{tabular}
\end{center}

\vspace{8mm}

\begin{table}[!htp]
        \renewcommand{\arraystretch}{1.5} % 仅影响此表的行高,行高设置为默认的1.5倍
	\centering
	\begin{tabular}{|c|c|c|c|c|}
		\hline
		\multirow{2}{*}{\Large \textbf{实验 (70分)}} & \Large 目标1 & \Large 目标2 & \Large 目标3 & \Large 得分 \\ \cline{2-5}
		                                    &             &             &             &            \\ \hline
		\multirow{2}{*}{\Large \textbf{报告 (30分)}} & \Large 代码 (10分) & \Large 结果 (10分) & \Large 格式 (10分) & \Large 得分 \\ \cline{2-5}
		                                    &             &             &             &            \\ \hline
		\multicolumn{2}{|c|}{\Large \textbf{批阅人}} & \multicolumn{2}{c|}{\Large \textbf{批阅日期}} & \Large \textbf{总得分} \\ \hline
		\multicolumn{2}{|c|}{}              & \multicolumn{2}{c|}{}              &            \\ \cline{1-2} \cline{3-5}
	\end{tabular}
\end{table}

\vspace{6mm}

\centerline{\kaishu\Large{上海大学\ 计算机工程与科学学院}}


%%%%%%
\newpage
\hypersetup{linkcolor=black,citecolor=black}
\tableofcontents
%%%%%%

%%%%%%%%%%%%%%%%%%%%%%%%%%%%%
%% The Content
%%%%%%%%%%%%%%%%%%%%%%%%%%%%%

\newpage
\setcounter{page}{1}


\section{选修实验8-2 基于Llama 2实现聊天机器人}

\subsection{实验目的}
\begin{itemize}
\item[$\bullet$] 熟悉Llama 2大语言模型的算法原理,掌握在DLP平台上移植优化聊天机器人的方法和流程,具体包括:
\item[$\bullet$] 本实验旨在向学生介绍基于大型语言模型 Llama 2 构建聊天机器人的基本原理和操作步骤,深入理解其在对话生成中的应用;
\item[$\bullet$] 学生将深入了解关键概念,包括模型加载和适配、对话模板的应用、文本生成的基本流程、模型推理的关键步骤,以及使用命令行进行聊天的具体过程;
\item[$\bullet$] 学生将了解如何在 DLP 平台上部署模型,实现人机快速聊天应用,为将来在实际场景中应用模型提供实用经验。
\end{itemize}


\subsection{实验环境}
\begin{itemize}
    \item[$\bullet$] 硬件平台:DLP云平台环境。
    \item[$\bullet$] 软件环境:编程框架Pytorch1.13.1、CNNL高性能AI运算库,CNRT 运行时库,以及 python 环境及相关的扩展库。
\end{itemize}

\subsection{评分标准}
\begin{itemize}
\item[$\bullet$] 60分标准:能够正确实现模型适配模块、对话模板模块,其中无论是否提供对话模板都能正确地实现程序。
\item[$\bullet$] 80分标准:在60分标准基础上,能够正确实现文本生成模块。
\item[$\bullet$] 100分标准:在80分标准基础上,能够采用Simple和Rich两种聊天模式与机器人进行交互,实现FastChat模型推断模块,可以完成聊天应用。
\end{itemize}

\subsection{实验内容及步骤}
\subsubsection{补全fastchat/utils.py}
\par 该文件主要包含了一些辅助工具函数和类,用于日志记录,重定向标准输出和错误输出,
获取GPU信息等。
\par 作业主要任务是补全get\_gpu\_memory()函数,代码补全如下:

\begin{lstlisting}[language=python, caption={utils.py}]
def get_gpu_memory(max_gpus=None):
    """Get available memory for each GPU."""
    #TODO: 存储每个GPU的可用内存信息
    gpu_memory = []
    #TODO: 获取MLU设备的数量,如果未指定最大GPU数(max_gpus为None),则使用所有可用设备;否则,使用max_gpus和实际设备数量中的较小值
    num_gpus = (
        min(max_gpus, torch.mlu.device_count())
        if max_gpus is not None
        else torch.mlu.device_count()
    )

    for gpu_id in range(num_gpus):
        #TODO: 将当前MLU设备设置为gpu_id
        torch.mlu.set_device(f"mlu:{gpu_id}")
        #TODO: 获取当前MLU设备
        device = torch.mlu.current_device()
        #TODO: 获取当前MLU设备的属性
        gpu_properties = torch.mlu.get_device_properties(device)
        #TODO: 获取总内存,单位转换为GB
        total_memory = gpu_properties.total_memory / (1024 ** 3) # GB
        #TODO: 获取已分配内存,单位转换为GB
        allocated_memory = torch.mlu.memory_allocated() / (1024 ** 3) # GB
        #TODO: 计算可用内存,单位转换为GB
        available_memory = total_memory - allocated_memory
        #TODO:将可用内存信息添加到列表中
        gpu_memory.append(available_memory)
    return gpu_memory
\end{lstlisting}

\subsubsection{补全fastchat/model/model\_adapter.py}
\par 该文件主要用于模型适配器的注册和管理,它定义了一个基础模型适配器类BaseAdapter及其多个子类。
每个子类正对不同的模型类型实现了特定的加载和匹配逻辑。文件还包含了用于注册模型适配器的函数、
加载模型的函数以及获取对话模板的函数。通过这些适配器,可以根据模型路径动态选择和加载合适的模型和分词器。
\par 作业主要任务是补全load\_model()函数和get\_conversation\_template()函数。
\par 由于篇幅限制,这里只展示部分补全的代码

\begin{lstlisting}[language=python, caption={load\_model()函数}]
def load_model(
    model_path: str,
    device: str,
    num_gpus: int,
    max_gpu_memory: Optional[str] = None,
    load_8bit: bool = False,
    cpu_offloading: bool = False,
    debug: bool = False,
):
    """Load a model from Hugging Face."""

    # TODO: 处理设备映射,调用函数,检查并更新CPU offloading的配置
    cpu_offloading = raise_warning_for_incompatible_cpu_offloading_configuration(
        device, load_8bit, cpu_offloading
    )
    # TODO: 如果设备是CPU
    if device == "cpu":
        kwargs = {"torch_dtype": torch.float32}
    # TODO: 如果设备是MLU
    elif device == "mlu":
        kwargs = {"torch_dtype": torch.float16}
        if num_gpus != 1:
            kwargs["device_map"] = "auto"
            if max_gpu_memory is None:
                kwargs["device_map"] = (
                    "sequential"  # This is important for not the same VRAM sizes
                )
                # TODO:获取每个GPU的可用内存
                available_gpu_memory = get_gpu_memory(num_gpus)
                kwargs["max_memory"] = {
                    i: str(int(available_gpu_memory[i] * 0.85)) + "GiB"
                    for i in range(num_gpus)
                }
            else:
                kwargs["max_memory"] = {i: max_gpu_memory for i in range(num_gpus)}
    # TODO: 如果设备是mps
    elif device == "mps":
        kwargs = {"torch_dtype": torch.float16}
        # Avoid bugs in mps backend by not using in-place operations.
        replace_llama_attn_with_non_inplace_operations()
    else:
        raise ValueError(f"Invalid device: {device}")

    # TODO: 如果启用了CPU offloading
    if cpu_offloading:
        # raises an error on incompatible platforms
        from transformers import BitsAndBytesConfig

        if "max_memory" in kwargs:
            kwargs["max_memory"]["cpu"] = (
                str(math.floor(psutil.virtual_memory().available / 2**20)) + "Mib"
            )
        kwargs["quantization_config"] = BitsAndBytesConfig(
            load_in_8bit_fp32_cpu_offload=cpu_offloading
        )
        kwargs["load_in_8bit"] = load_8bit
    # TODO: 如果启用了8位量化但未启用CPU offloading
    elif load_8bit:
        if num_gpus != 1:
            warnings.warn(
                "8-bit quantization is not supported for multi-gpu inference."
            )
        else:
            # TODO:调用压缩模型的函数
            return load_compress_model(model_path, device=device, **kwargs)

    # TODO:调用函数,根据给定的模型路径获取适配器,遍历已注册的模型适配器列表,返回第一个匹配的适配器实例。
    adapter = get_model_adapter(model_path)
    # TODO:使用适配器adapter加载模型和分词器
    model, tokenizer = adapter.load_model(model_path, kwargs)

    if (device == "mlu" and num_gpus == 1 and not cpu_offloading) or device == "mps":
        model.to(device)

    if debug:
        print(model)
    print("MODEL ADAPTER PASS!")
    return model, tokenizer
\end{lstlisting}

\subsubsection{补全fastchat/conversation.py}
\par 该文件主要定义了对话提示模板和对话类。它包含了不同分隔符样式的枚举类和用于存储对话历史的类。
\par 文件还定义了一些全局对话模板,并提供了注册和获取对话模板的函数。
通过这些模板,可以方便地管理和生成不同风格的对话。
\par 作业主要为补全get\_prompt()函数和append\_message()函数。
\par 由于篇幅限制,且此部分代码不是重点,补全代码不在此展示,详细内容见代码文件。

\subsubsection{补全fastchat/model/chatglm\_model.py}
\par 该文件主要包含两个函数。
\par stream\_chat\_token\_num()函数用于计算给定对话历史和当前问题的token数量。
\par chatglm\_generate\_stream()函数使用给定的模型和分词器,根据输入参数生成文本流,支持多轮对话,
并在生成过程中返回生成的文本和相关的token数量。
\par 同理篇幅限制,补全详情见代码文件。

\subsubsection{补全fastchat/serve/inference.py}
\par 该文件是实现推理的核心功能,主要用于FastChat模型的推理,它定义了生成文本流的函数generate\_stram,
并导入了chatglm\_generate\_stream函数,用于根据输入参数生成对话文本。
\par chatloop负责加载模型和分词器,管理对话循环,并调用相应的生成函数生成和输出对话文本。
\par 这里需要注意的是,在generate\_stream函数中,要及时关闭梯度计算,否则会导致内存泄漏。
\par 由于此部分是实现推理的核心功能,我将详细展示各补全的接口函数。

\begin{lstlisting}[language=python, caption={generate\_stream()函数}]
def generate_stream(
    model, tokenizer, params, device, context_len=2048, stream_interval=2
):
    #TODO:# 关闭梯度计算
    with torch.no_grad():
        prompt = params["prompt"]
        #TODO: 获取prompt的长度
        len_prompt = len(prompt) if isinstance(prompt, str) else sum(len(s) for s in prompt)
        #TODO: 从参数中获取生成文本时的temperature(控制文本生成的多样性),如果参数中未指定,则默认为 1.0。
        temperature = float(params.get("temperature", 1.0))
        #TODO: 从参数中获取生成文本时的 repetition penalty(抑制文本中重复的程度),如果参数中未指定,则默认为 1.0。
        repetition_penalty = float(params.get("repetition_penalty", 1.0))
        #TODO:从参数中获取生成文本时的 top_p(控制生成文本的多样性),如果参数中未指定,则默认为 1.0。
        top_p = float(params.get("top_p", 1.0))
        #TODO: 从输入的参数中获取 top_k 的值,如果参数中没有设置,则默认为 -1。
        top_k = int(params.get("top_k", -1))  # -1 means disable
        #TODO 从参数中获取生成文本时的max_new_tokens数,如果参数中未指定,则默认为 256。
        max_new_tokens = int(params.get("max_new_tokens", 256))
        #TODO: 从参数中获取 stop_str,如果未设置,则默认为 None
        stop_str = params.get("stop", None)
        #TODO: 从参数中获取 echo,如果未设置,默认为 True。并将其转换为布尔值。
        echo = bool(params.get("echo", True))
        stop_token_ids = params.get("stop_token_ids", None) or []
        #TODO:将文本生成停止的标记添加到已有的停止标记列表stop_token_ids中。
        if stop_str and isinstance(stop_str, str):
            # 这里仅取第一个token id作为示例,可根据需要调整
            stop_token_ids.append(tokenizer.encode(stop_str, add_special_tokens=False)[0])

        #TODO: 创建一个 logits 处理器列表
        logits_processor = prepare_logits_processor(temperature, repetition_penalty, top_p, top_k)

        #TODO: 使用tokenizer将输入文本转换为模型可接受的输入张量
        input_ids = tokenizer.encode(prompt, add_special_tokens=False)
        #TODO: 记录输入文本的长度
        input_echo_len = len(input_ids)
        #TODO: 创建一个名为 output_ids 的列表,其初始值等于 input_ids 列表的内容
        output_ids = input_ids.copy()

        if model.config.is_encoder_decoder:
            max_src_len = context_len
        else:
            max_src_len = context_len - max_new_tokens - 8

        #TODO: 截取源文本的最后一部分,以适应模型的上下文长度
        if len(input_ids) > max_src_len:
            input_ids = input_ids[-max_src_len:]

        if model.config.is_encoder_decoder:
            encoder_output = model.encoder(
                input_ids=torch.as_tensor([input_ids], device=device)
            )[0]
            start_ids = torch.as_tensor(
                [[model.generation_config.decoder_start_token_id]],
                dtype=torch.int64,
                device=device,
            )

        past_key_values = out = None
        for i in range(max_new_tokens):
            if i == 0:
                if model.config.is_encoder_decoder:
                    #TODO: 使用解码器对起始标记进行处理
                    out = model.decoder(
                        input_ids=start_ids,
                        encoder_hidden_states=encoder_output,
                        use_cache=True,
                    )
                    logits = model.lm_head(out[0])
                else:
                    #TODO: 非编码解码器类型,直接使用模型进行处理
                    out = model(
                        input_ids=torch.as_tensor([input_ids], device=device),
                        use_cache=True,
                    )
                    logits = out.logits
                #TODO: 记录过去的键值
                past_key_values = out.past_key_values
            else:
                if model.config.is_encoder_decoder:
                    #TODO: 使用解码器对当前标记进行处理
                    out = model.decoder(
                        input_ids=torch.as_tensor([[output_ids[-1]]], device=device),
                        encoder_hidden_states=encoder_output,
                        use_cache=True,
                        past_key_values=past_key_values,
                    )
                    #TODO: 获取logits(预测的标记分布)
                    logits = model.lm_head(out[0])
                else:
                    out = model(
                        #TODO:非编码解码器类型,直接使用模型对当前标记进行处理
                        input_ids=torch.as_tensor([[output_ids[-1]]], device=device),
                        use_cache=True,
                        past_key_values=past_key_values,
                    )
                    logits = out.logits
                #TODO: 更新过去的键值
                past_key_values = out.past_key_values

            if logits_processor:
                if repetition_penalty > 1.0:
                    tmp_output_ids = torch.as_tensor([output_ids], device=logits.device)
                else:
                    tmp_output_ids = None
                #TODO: 使用logits_processor处理最后一个标记的logits
                last_token_logits = logits_processor(tmp_output_ids, logits[:, -1, :])
            else:
                #TODO: 没有logits处理器,直接获取最后一个标记的logits
                last_token_logits = logits[:, -1, :]

            if device == "mps":
                # Switch to CPU by avoiding some bugs in mps backend.
                last_token_logits = last_token_logits.float().to("cpu")

            if temperature < 1e-5 or top_p < 1e-8:  # greedy
                #TODO:# 通过argmax获取概率最高的标记索引,并将其转换为整数类型。
                token = int(torch.argmax(last_token_logits, dim=-1).item())
            else:
                #TODO: 使用softmax将概率分布归一化
                probs = torch.softmax(last_token_logits / temperature, dim=-1)
                #TODO: 使用torch.multinomial函数根据概率分布采样生成标记,并转换为整数类型。
                token = int(torch.multinomial(probs, num_samples=1).item())
            
            # 将生成的token添加到输出序列output_ids中
            output_ids.append(token)

            if token in stop_token_ids:
                stopped = True
            else:
                stopped = False

            #TODO:判断是否达到生成结果的间隔、已完成生成最大标记数、或者已经停止生成 
            if (i % stream_interval == 0) or (i == max_new_tokens - 1) or stopped:
                if echo:
                    tmp_output_ids = output_ids
                    rfind_start = len_prompt
                else:
                    tmp_output_ids = output_ids[input_echo_len:]
                    rfind_start = 0

                output = tokenizer.decode(
                    tmp_output_ids,
                    skip_special_tokens=True,
                    spaces_between_special_tokens=False,
                )
                if stop_str:
                    if isinstance(stop_str, str):
                        #TODO:在输出字符串中从右向左搜索停止标记的位置,rfind_start参数指定了搜索的起始位置
                        pos = output.rfind(stop_str, rfind_start)
                        if pos != -1:
                            #TODO: 将输出字符串截断,仅保留停止标记位置之前的部分
                            output = output[:pos]
                            stopped = True
                    elif isinstance(stop_str, Iterable):
                        for each_stop in stop_str:
                            #TODO: 从指定位置(rfind_start)向前搜索每个停止标记在输出字符串中的最后出现位置
                            pos = output.rfind(each_stop, rfind_start)
                            if pos != -1:
                                #TODO: 将输出字符串截断,仅保留停止标记位置之前的部分
                                output = output[:pos]
                                stopped = True
                                break
                    else:
                        raise ValueError("Invalid stop field type.")

                    yield {
                        "text": output,
                        "usage": {
                            "prompt_tokens": input_echo_len,
                            "completion_tokens": i,
                            "total_tokens": input_echo_len + i,
                        },
                        "finish_reason": None,
                    }

            if stopped:
                break

        # finish stream event, which contains finish reason
        if i == max_new_tokens - 1:
            finish_reason = "length"
        elif stopped:
            finish_reason = "stop"
        else:
            finish_reason = None

        #TODO: 返回生成的文本、使用情况和结束原因的字典
        yield {
            "text": tokenizer.decode(output_ids[input_echo_len:], skip_special_tokens=True, spaces_between_special_tokens=False),
            "usage": {
                "prompt_tokens": input_echo_len,
                "completion_tokens": i if not stopped else i+1,
                "total_tokens": input_echo_len + (i if not stopped else i+1),
            },
            "finish_reason": finish_reason,
        }

        # clean
        del past_key_values, out
        gc.collect()
        #TODO:释放MLU设备上的缓存空间
        torch.mlu.empty_cache()
\end{lstlisting}

\par 该函数用于在生成模型上执行流式文本生成。其流程包括:
\par 从输入参数中提取生成配置(如温度、重复惩罚、top\_p、top\_k、最大token数、停止字符串/标记等)。
\par 利用tokenizer将提示文本转为模型输入,截断以适应模型的最大上下文长度。
\par 根据模型类型(编码器-解码器或普通生成模型),分别调用相应的模型接口进行首次生成和后续token生成,
同时利用past\_key\_values缓存先前的计算结果以提高效率。
\par 使用logits处理器和概率采样方法选取下一个token,将其拼接到输出token列表中。
\par 根据设定的输出间隔、达到最大token或遇到停止标记,实时解码并返回当前生成的文本,同时记录token使用情况。
\par 最终,当生成完成或触发停止条件时,返回最终的生成结果以及相关的使用信息,并清理缓存。
\par 这里注意,最后的清理缓存,toch\_mlu中不存在相关缓存清理函数(曾困扰了我很久)。且要及时关闭梯度计算,否则会爆显存。

\begin{lstlisting}[language=python, caption={chat\_loop()函数}]
def chat_loop(
    model_path: str,
    device: str,
    num_gpus: int,
    max_gpu_memory: str,
    load_8bit: bool,
    cpu_offloading: bool,
    conv_template: Optional[str],
    temperature: float,
    max_new_tokens: int,
    chatio: ChatIO,
    debug: bool,
):
    #TODO:调用load_model函数加model和tokenizer
    model, tokenizer = load_model(
        model_path,
        device=device,
        num_gpus=num_gpus,
        max_gpu_memory=max_gpu_memory,
        load_8bit=load_8bit,
        cpu_offloading=cpu_offloading,
    )
    is_chatglm = "chatglm" in str(type(model)).lower()

    #TODO: 如果提供了对话模板,使用提供的模板,调用get_conv_template创建会话对象
    if conv_template:
        conv = get_conv_template(conv_template)
    else:
        #TODO:否则使用默认的对话模板,调用get_conversation_template创建会话对象
        conv = get_conversation_template("default")
    print("GENERATE STEAM PASS!")   
    while True:
        try:
            #TODO: 尝试获取用户输入,传入 conv.roles[0] 作为角色标识
            inp = chatio.prompt_for_input(conv.roles[0])
        except EOFError:
            inp = ""
        if not inp:
            print("exit...")
            break

        conv.append_message(conv.roles[0], inp)
        conv.append_message(conv.roles[1], None)

        if is_chatglm:
            generate_stream_func = chatglm_generate_stream
            prompt = conv.messages[conv.offset :]
        else:
            generate_stream_func = generate_stream
            prompt = conv.get_prompt()

        gen_params = {
            "model": model_path,
            "prompt": prompt,
            "temperature": temperature,
            "max_new_tokens": max_new_tokens,
            "stop": conv.stop_str,
            "stop_token_ids": conv.stop_token_ids,
            "echo": False,
        }

        #TODO: 获取机器对话输出,传入 conv.roles[1] 作为角色标识
        chatio.prompt_for_output(conv.roles[1])
        output_stream = generate_stream_func(model, tokenizer, gen_params, device)
        #TODO:# 输出对话的流式输出
        outputs = chatio.stream_output(output_stream)
 
        # NOTE: strip is important to align with the training data.
        conv.messages[-1][-1] = outputs.strip()
        if debug:
            print("\n", {"prompt": prompt, "outputs": outputs}, "\n")
        print("FASTCHAT INFERENCE PASS!")
\end{lstlisting}

\par 该函数是实现多轮交互对话的关键功能。
\par 通过加载模型和tokenizer,根据传入的对话模板参数创建会话对象,(可使用自定义或默认模板)。
在循环中,每次获取用户输入,并将用户信息添加到对话记录中。根据模型类型选择生成函数(chatglm或普通生成),
然后构造生成参数,并调用模型进行流式文本生成,然后实时显示生成结果。

\subsubsection{补全fastchat/serve/cli.py}
\par 该文件提供了一个命令行界面与模型进行对话。还定义了两个类,SimpleChatIO和RichChatIO,分别用于简单对话和富文本对话,都继承自ChatIO基类。
\par 主函数main解析命令行参数,选择合适的输入输出类,并调用chat\_loop函数启动聊天循环,实现与模型的交互。
\par 其中还含有命令行参数的定义和选项。
\par 该文件是对话程序的主逻辑,我将详细展示补全代码:

\begin{lstlisting}[language=python, caption={SimpleChatIO类}]
    #TODO:构建SimpleChatIO类,它继承自ChatIO类
class SimpleChatIO(ChatIO):
    def prompt_for_input(self, role) -> str:
        return input(f"{role}: ")

    def prompt_for_output(self, role: str):
        print(f"{role}: ", end="", flush=True)

    def stream_output(self, output_stream):
        pre = 0
        for outputs in output_stream:
            output_text = outputs["text"]
            #TODO: 移除文本两端的空白字符,然后按空格分割
            output_text = output_text.strip().split(" ")
            #TODO: 获取处理后的文本中单词的数量
            now = len(output_text)
            if now > pre:
                # 输出不同于前次的部分,其中,新增的内容以空格分隔的形式显示,确保每次新增的内容都在同一行,并及时刷新输出缓冲区。
                print(" ".join(output_text[pre:now]), end=" ", flush=True)
                pre = now
        print(" ".join(output_text[pre:]), flush=True)
        print("CHATIO PASS!")
        return " ".join(output_text)
\end{lstlisting}

\par 该类继承自inference.py文件中的ChatIO类,用于实现简单的命令行交互输入输出。
主要功能包括prompt\_for\_input(),根据指定角色提示用户输入,并返回输入内容;
prompt\_for\_output(),在命令行中打印角色标识,准备输出生成的回答;
stream\_output(),流式处理生成器返回的文本,在处理过程中,函数会对文本空白字符去除和按空格分割,
然后只输出新生成的部分,确保输出及时刷新,直至完成整个生成过程,最终函数返回完整的输出文本。

\begin{lstlisting}[language=python, caption={RichChatIO类}]
    #TODO:构建RichChatIO类,它继承自ChatIO类
class RichChatIO(ChatIO):
    def __init__(self):
        #TODO: 创建PromptSession实例,用于获取用户输入,并设置输入历史记录
        self._prompt_session = PromptSession(history=InMemoryHistory())
        #TODO: 创建自动补全器,用于用户输入的自动完成
        self._completer = WordCompleter(words=["!exit", "!reset"], pattern=re.compile(r"$"))
        #TODO:创建Console 实例,在命令行界面中以更丰富的样式显示文本
        self._console = Console()

    def prompt_for_input(self, role) -> str:
        self._console.print(f"[bold]{role}:")
        # TODO(suquark): multiline input has some issues. fix it later.
        prompt_input = self._prompt_session.prompt(
            completer=self._completer,
            multiline=False,
            #TODO:启用自动建议功能
            auto_suggest=AutoSuggestFromHistory(),
            key_bindings=None,
        )
        self._console.print()
        return prompt_input

    def prompt_for_output(self, role: str):
        self._console.print(f"[bold]{role}:")

    def stream_output(self, output_stream):
        """Stream output from a role."""
        # TODO(suquark): the console flickers when there is a code block
        #  above it. We need to cut off "live" when a code block is done.

        # Create a Live context for updating the console output

        #TODO: 创建Live上下文管理器,用于实现在命令行中实时更新显示。其中,需要指定要在其上执行实时更新的console实例,每秒刷新的次数设为4
        with Live(console=self._console, refresh_per_second=4) as live:
            # Read lines from the stream
            for outputs in output_stream:
                if not outputs:
                    continue
                # 如果输出是字典,则获取字典中的"text"键对应的值,否则直接获取输出
                if isinstance(outputs, dict):
                    text = outputs.get("text", "")
                else:
                    text = outputs
    
                lines = []
                for line in text.splitlines():
                    if "### Human" in line:
                        continue
                    lines.append(line)
                    if line.startswith("```"):
                        # Code block marker - do not add trailing spaces, as it would
                        #  break the syntax highlighting
                        lines.append("\n")
                    else:
                        lines.append("  \n")
                markdown = Markdown("".join(lines))
                #TODO: 将渲染后的 markdown 文本实时更新到控制台
                live.update(markdown)
        self._console.print()
        print("CHATIO PASS!")
        return text
\end{lstlisting}

\par 该类继承自inference.py文件中的ChatIO类,用于实现富文本的命令行交互输入输出。
同样包括在控制台以加粗样式输出角色标识,并使用PromptSession获取用户输入,同时启用历史自动建议。
在stream\_output()函数中,使用Rich的Live上下文管理器实现实时刷新,将流式获取的输出文本以Markdown格式渲染,
然后实时更新显示,确保生成过程在终端平滑显示。最终稿返回输出文本。提供了更加美观的交互界面。

\begin{lstlisting}[language=python, caption={主函数部分}]
def main(args):
    if args.gpus:
        if len(args.gpus.split(",")) < args.num_gpus:
            raise ValueError(
                f"Larger --num-gpus ({args.num_gpus}) than --gpus {args.gpus}!"
            )
        os.environ["CUDA_VISIBLE_DEVICES"] = args.gpus

    if args.style == "simple":
        chatio = SimpleChatIO()
    elif args.style == "rich":
        chatio = RichChatIO()
    else:
        raise ValueError(f"Invalid style for console: {args.style}")
    try:
        #TODO:调用 chat_loop 函数,启动聊天循环
        chat_loop(
            args.model_path,
            args.device,
            args.num_gpus,
            args.max_gpu_memory,
            args.load_8bit,
            args.cpu_offloading,
            args.conv_template,
            args.temperature,
            args.max_new_tokens,
            chatio,
            args.debug,
        )
    except KeyboardInterrupt:
        print("exit...")


if __name__ == "__main__":
    #TODO: 创建一个ArgumentParser对象,用于解析命令行参数
    parser = argparse.ArgumentParser()
    #TODO: 向ArgumentParser对象中添加模型相关的参数,这些参数由add_model_args函数定义
    add_model_args(parser)
    parser.add_argument(
        "--conv-template", type=str, default=None, help="Conversation prompt template."
    )
    parser.add_argument("--temperature", type=float, default=0.7)
    parser.add_argument("--max-new-tokens", type=int, default=512)

    #TODO:试一下simple模式,同时也试一下rich模式。
    parser.add_argument(
        "--style",
        type=str,
        default="simple",
        choices=["simple", "rich"],
        help="Display style.",
    )
    parser.add_argument("--debug", action="store_true", help="Print debug information")
    args = parser.parse_args()
    main(args)
\end{lstlisting}

\par 这段主函数部分,实现了命令行应用程序的入口。主要包括:
根据命令行参数配置GPU,根据用户指定的显示风格(simple或rich)实例化对应的ChatIO类,以确定交互界面的显示方式。
解析并设置模型相关与对话生成的参数,例如模型路径,对话模板,温度,最大token数等。
调用inference.py文件中chat\_loop函数,启动多轮对话循环,进行用户输入,文本生成与输出展示。
同时有捕获异常,实现用户中断时的退出。
\par 整体上这一部分负责初始化应用环境,解析参数,并启动聊天交互过程。同时也是测试环境中,脚本测试的对象文件。

\par 在开发环境中,通过运行当前目录下的的run\_scripts文件夹中的run\_mlu\_infer.sh脚本,可以对cli.py进行测试,从而测试整个模型推理效果。
\par 脚本命令如下:

\begin{lstlisting}[language=bash, caption={run\_completion.sh}, label={code:shell}]
set -e

model_path=/workspace/model/favorite/large-scale-models/model-v1/Llama-2-7b-hf
echo "********************************"
echo "**命令行终端形式 仅支持单轮对话**"
echo "********************************"
# 单卡
# 配置板卡
export MLU_VISIBLE_DEVICES=0
python3 -m fastchat.serve.cli --model-path ${model_path} --style=rich
\end{lstlisting}

在脚本命令后输入相应参数可实现多种功能,如调试模式,选择富文本或简单文本等。
\par 运行结果如图\ref{8-2-1}-\ref{8-2-3}所示:

\begin{figure}[!htbp]
    \centering
    \includegraphics[width=1\textwidth]{image/8-2/8-2-1.png}
    \caption{多轮对话结果展示1}
    \label{8-2-1}
\end{figure}

\begin{figure}[!htbp]
    \centering
    \includegraphics[width=1\textwidth]{image/8-2/8-2-2.png}
    \caption{多轮对话结果展示2}
    \label{8-2-2}
\end{figure}

\begin{figure}[!htbp]
    \centering
    \includegraphics[width=1\textwidth]{image/8-2/8-2-3.png}
    \caption{多轮对话结果展示3}
    \label{8-2-3}
\end{figure}

\par 在测试中,我询问了AI,“你是谁”,“询问了质能方程”,“询问了牛顿第二定律”。
AI均能正确回答问题。
\par 可见,正确实现了对话功能,可以与模型进行多轮对话交互。
\par 之前在希冀平台提交的结果中,最后两项chatio和fastchat\_inference始终未通过。后来经过助教通知,了解到,希冀平台的评测机机制中,没法处理多轮对话交互的情况。解决办法是将inference.py中的chat\_loop函数中的while(true)循环逻辑移除,并为inp赋值为一个固定的内容,即可。修改的代码部分如下,见注释部分和inp变量:

\begin{lstlisting}[language=python, caption={修改交互部分}]
def chat_loop(
    model_path: str,
    device: str,
    num_gpus: int,
    max_gpu_memory: str,
    load_8bit: bool,
    cpu_offloading: bool,
    conv_template: Optional[str],
    temperature: float,
    max_new_tokens: int,
    chatio: ChatIO,
    debug: bool,
):
    # TODO:调用load_model函数加model和tokenizer
    model, tokenizer = load_model(
        model_path,
        device=device,
        num_gpus=num_gpus,
        max_gpu_memory=max_gpu_memory,
        load_8bit=load_8bit,
        cpu_offloading=cpu_offloading,
    )
    is_chatglm = "chatglm" in str(type(model)).lower()

    # TODO: 如果提供了对话模板,使用提供的模板,调用get_conv_template创建会话对象
    if conv_template:
        conv = get_conv_template(conv_template)
    else:
        # TODO:否则使用默认的对话模板,调用get_conversation_template创建会话对象
        conv = get_conversation_template("default")
    print("GENERATE STEAM PASS!")
    # while True:
    #     try:
    #         # TODO: 尝试获取用户输入,传入 conv.roles[0] 作为角色标识
    #         inp = chatio.prompt_for_input(conv.roles[0])
    #     except EOFError:
    #         inp = ""
    #     if not inp:
    #         print("exit...")
    #         break
    inp = "hello, who are you?"
    conv.append_message(conv.roles[0], inp)
    conv.append_message(conv.roles[1], None)

    if is_chatglm:
        generate_stream_func = chatglm_generate_stream
        prompt = conv.messages[conv.offset :]
    else:
        generate_stream_func = generate_stream
        prompt = conv.get_prompt()

    gen_params = {
        "model": model_path,
        "prompt": prompt,
        "temperature": temperature,
        "max_new_tokens": max_new_tokens,
        "stop": conv.stop_str,
        "stop_token_ids": conv.stop_token_ids,
        "echo": False,
    }

    # TODO: 获取机器对话输出,传入 conv.roles[1] 作为角色标识
    chatio.prompt_for_output(conv.roles[1])
    output_stream = generate_stream_func(model, tokenizer, gen_params, device)
    # TODO:# 输出对话的流式输出
    outputs = chatio.stream_output(output_stream)

    # NOTE: strip is important to align with the training data.
    conv.messages[-1][-1] = outputs.strip()
    if debug:
        print("\n", {"prompt": prompt, "outputs": outputs}, "\n")
    print("FASTCHAT INFERENCE PASS!")
\end{lstlisting}

\par 最后,希冀平台评测结果如图\ref{8-2}:

\begin{figure}[!htbp]
    \centering
    \includegraphics[width=1\textwidth]{image/8-2/8-2-final.png}
    \caption{8-2测试结果}
    \label{8-2}
\end{figure}

\par 可见,成功通过了所有测试点,实验完成。

\subsection{实验总结}

\subsubsection{实验过程回顾}
\par 实验初期,面对复杂的实验环境和众多的代码文件,我感到有些无从下手。
但在逐步梳理实验步骤后,我开始按照计划补全各个文件中的代码。
我一边查阅资料学习相关知识,一边仔细推敲每一行代码的逻辑,确保其功能的正确实现。
\subsubsection{实验成果与不足}
\par 经过不懈努力,我成功实现了基于 Llama 2 的聊天机器人,并通过命令行界面与模型进行了多轮对话交互。
在开发环境中测试时,结果显示了最终的 pass 信息,表明实验功能基本实现。
然而,在希冀平台的判题系统中,却未通过最后两项 chatio 和 fastchat\_inference 测试,
这让我感到有些遗憾。尽管我尝试了多种方法进行调试,但仍未找到具体原因,
可能是接口功能未被测试捕捉到,也可能是代码本身存在一些隐藏的 BUG。
\subsubsection{实验收获}
\paragraph{技术能力提升:}
通过本次实验,我深入学习了 Llama 2 模型的相关知识,掌握了在 DLP 平台上进行模型移植优化的方法,熟悉了 Pytorch 框架下模型的加载、适配、推理等操作流程,提升了自己在智能计算领域的技术能力。
\paragraph{问题解决能力锻炼:}
在实验过程中遇到的各种问题,促使我不断查阅资料、分析代码、调试程序,锻炼了我的问题解决能力。面对复杂的代码和未知的错误,我学会了保持冷静,从不同角度思考问题,逐步排查并解决问题。
\paragraph{理论与实践结合:}
本次实验将课堂上学到的理论知识与实际操作紧密结合,让我更加深刻地理解了智能计算系统的原理和应用。通过亲自动手实现聊天机器人,我体会到了理论指导实践、实践检验理论的重要性。
\subsubsection{未来展望}
虽然本次实验取得了一定的成果,但也暴露出了一些不足之处。
在今后的学习和实践中,我将继续深入研究智能计算领域的相关技术,
不断提升自己的技术水平。同时,我也会更加注重细节,努力提高代码的质量和稳定性,
避免类似的问题再次出现。此外,我还希望能够将所学知识应用到更多的实际项目中,
为智能计算技术的发展贡献自己的一份力量。
\end{document}
