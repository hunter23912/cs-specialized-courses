%!TEX program = xelatex
% 完整编译: xelatex -> biber/bibtex -> xelatex -> xelatex
\documentclass[lang=cn,a4paper]{shu-lab-report}


% 本文档命令
\usepackage{array}
\newcommand{\ccr}[1]{\makecell{{\color{#1}\rule{1cm}{1cm}}}}
\renewcommand{\lstlistingname}{代码}




\begin{document}

%%%%%%%%%%%%%%%%%%%%%%%%%%%%%
%% The Cover Page
%%%%%%%%%%%%%%%%%%%%%%%%%%%%%

\thispagestyle{empty}

~\\


\begin{figure}[!htbp]
    \centering
    \includegraphics[width=10.5cm]{image/shulogo.png}
\end{figure}


\vspace{12mm}

\centerline{\kaishu\Huge{2024-2025学年冬季学期}}

\vspace{6mm}

\centerline{\kaishu\Huge{《智能计算系统》(08696037)}}

\vspace{6mm}

\centerline{\kaishu\Huge{实验报告}}


\vspace{16mm}

\begin{center}
\renewcommand\arraystretch{1.5}
\begin{tabular}{r c}
    \makebox[8em][s]{\LARGE{姓名}}    & \LARGE{XXX}\\  \cmidrule(l){2-2} 
    \makebox[8em][s]{\LARGE{学号}}    & \LARGE{2212XXXX}\\  \cmidrule(l){2-2} 
    \makebox[8em][s]{\LARGE{实验名称}} & \LARGE{基于 Code Llama 实现代码生成}\\     \cmidrule(l){2-2} 
    \makebox[8em][s]{\LARGE{日期}}    & \LARGE{2025年X月X日}\\ \cmidrule(l){2-2}
\end{tabular}
\end{center}

\vspace{8mm}

\begin{table}[!htp]
        \renewcommand{\arraystretch}{1.5} % 仅影响此表的行高,行高设置为默认的1.5倍
	\centering
	\begin{tabular}{|c|c|c|c|c|}
		\hline
		\multirow{2}{*}{\Large \textbf{实验 (70分)}} & \Large 目标1 & \Large 目标2 & \Large 目标3 & \Large 得分 \\ \cline{2-5}
		                                    &             &             &             &            \\ \hline
		\multirow{2}{*}{\Large \textbf{报告 (30分)}} & \Large 代码 (10分) & \Large 结果 (10分) & \Large 格式 (10分) & \Large 得分 \\ \cline{2-5}
		                                    &             &             &             &            \\ \hline
		\multicolumn{2}{|c|}{\Large \textbf{批阅人}} & \multicolumn{2}{c|}{\Large \textbf{批阅日期}} & \Large \textbf{总得分} \\ \hline
		\multicolumn{2}{|c|}{}              & \multicolumn{2}{c|}{}              &            \\ \cline{1-2} \cline{3-5}
	\end{tabular}
\end{table}

\vspace{6mm}

\centerline{\kaishu\Large{上海大学\ 计算机工程与科学学院}}


%%%%%%
\newpage
\hypersetup{linkcolor=black,citecolor=black}
\tableofcontents
%%%%%%

%%%%%%%%%%%%%%%%%%%%%%%%%%%%%
%% The Content
%%%%%%%%%%%%%%%%%%%%%%%%%%%%%

\newpage
\setcounter{page}{1}


\section{实验1 多环境OpenMP程序的编译和运行}

\subsection{实验目的}
\begin{itemize}
\item[$\bullet$] 掌握OpenMP并行编程基础
\item[$\bullet$] 掌握在Linux平台上编译和运行OpenMP程序
\item[$\bullet$] 掌握在Windows平台上编译和运行OpenMP程序
\end{itemize}


\subsection{实验步骤}

\subsubsection{Linux下OpenMP程序的编译和运行}

\begin{itemize}
\item[$\bullet$]\textbf{编译环境配置}
\end{itemize}

\par GNU 编译器集合是一系列用于语言开发的编译器和库的集合,包括: C、 C++、 Objective-C、Fortran、Ada和Go等编程语言。很多开源项目,包括 Linux\ Kernel和 GNU 工具,都是使用 GCC 进行编译的。
\par 默认的 Ubuntu 软件源包含了一个软件包组,名称为 "build-essential",它包含了 GNU 编辑器集合,GNU 调试器,和其他编译软件所必需的开发库和工具。
想要安装开发工具软件包,以拥有 sudo 权限用户身份或者 root 身份运行下面的命令:
\begin{lstlisting}[language=C, caption={安装命令}, label={code:install}]
sudo apt-get update
sudo apt-get install build-essential
\end{lstlisting}
\par 这个命令将会安装一系列软件包,包括GCC、G++和Make。安装完成后,即可编译运行程序。

\begin{itemize}
\item[$\bullet$]\textbf{编译与运行}
\end{itemize}
\par 编写串行、并行但代码中不设置线程数(即默认线程),以及代码中设置8个线程的HelloWorld程序,并用下述命令编译程序。
\begin{lstlisting}[language=C, caption={编译命令}, label={code:compile}]
gcc -fopenmp -O2 -o helloomp.out helloworld.c
\end{lstlisting}
\par 编译完成后,输入./helloomp.out即可运行程序,运行结果如下所示:

\begin{figure}[!htbp]
    \centering
    \includegraphics[width=0.8\textwidth]{image/shulogo.png}
    \caption{Linux下helloworld.c运行结果}
    \label{fig:Linux_OMP}
\end{figure}

\end{document}
